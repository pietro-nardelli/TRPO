\documentclass[11pt]{article}
\usepackage{amsmath, amssymb, amscd, amsthm, amsfonts}
\usepackage{graphicx}
\usepackage{subcaption}
\graphicspath{ {./Images/} }
\usepackage{hyperref}

\oddsidemargin 0pt
\evensidemargin 0pt
\marginparwidth 40pt
\marginparsep 10pt
\topmargin -20pt
\headsep 10pt
\textheight 8.7in
\textwidth 6.65in
\linespread{1.2}

\title{Trust Region Policy Optimization - Implementation}
\author{Pietro Nardelli \\ \\ 
        Master's Degree in Artificial Intelligence and Robotics \\
        Department of Computer, Control and Management Engineering \\
        Sapienza University of Rome}

\date{January 2020}


\begin{document}

\maketitle

\section{Introduction}
Trust Region Policy Optimization is an algorithm that make several
approximations to a theoretical iterative procedure for optimizing policies with
guaranteeed monotonic improvement. Despite its approximations that deviate from
the theory, TRPO tends to give monotonic improvement, while little tuning of
hyperparameters. This algorithm is effective for optimizing large nonlinear
policies such as neural networks. The algorithm has been tested on two different
openAI gym environments. Gym library is a collection of test problems that can
be used to test reinforcement learning algorithms. 

\section{TRPO algorithm}\label{section-trpo}
Let $\pi$ denote a stochastic policy and let $\eta(\pi)$ denote its expected discounted reward. The
advantage function $A_{\pi} = Q_{\pi}(s, a) - V_{\pi}(s)$ is the difference between state-action
value function and value function. We can prove that any policy update $\pi \rightarrow \tilde{\pi}$
that has a nonnegative expected advantage at every states is guaranteeed to increase the policy
performance $\eta$, or leave it constant in the case that the expected advantage is zero everywhere.
However, in the approximate setting, it will typically be unavoidable, due to estimation and
approximation error, that there will be some state s for which the expected advantage is negative.
This is the idea at the base of the policy iteration algorithm that is a type of
minorization-maximization algorithm. The evolution of this kind of algorithm is called Trust Region
Policy Optimization (TRPO), which uses a constraint on the Kullback–Leibler (KL) divergence rather than a penalty to
robustly allow large updates.
The KL divergence is a measure of how one probability distribution $P$ is different from a second
probability distribution $Q$ that is a reference for the first one: $D_{KL}(P||Q)$.
In fact, supposing that we used the penalty coefficient recommended by theory, the step sizes would
be very small. One way to take larger steps in a robust way is to use a constraint on the KL
divergence between new policy and old policy (trust region constraint):
\begin{equation} 
        \begin{split}
        maximize_{\theta} \ L_{\theta_{old}}(\theta) 
        \\ \label{eq:1}
        subject \  to \ D_{KL_{max}}(\theta_{old}, \theta) \leq \delta
        \end{split}
\end{equation}

Knowing that $L$ is a local approximation to $\eta$ and
$D_{KL}(\theta|| \tilde{\theta}) := D_{KL}(\pi_{\theta}||\pi_{\tilde{\theta}})$ 
This problem imposes a constrain that the KL divergence is bounded at everyt point in the state
space. While it's motivated by the theory, this problem is impractical to solve due to large number
of constraints. Instead, we can used a heuristic approximation which considers the avarage KL
divergence $\bar{D_{KL}}$.
Our optimization problem in equation (\ref{eq:1}) is exactly equivalent to the following one, written
in terms of expectations, replacing some terms:
\begin{equation} 
        \begin{split}
        maximize_{\theta} \ E_t[\frac{\pi_{\theta}(a_t, s_t)}{\pi_{\theta_{old}}(a_t, s_t)}A_t]
        \\ \label{eq:2}
        subject \  to \ D_{KL_{max}}(\theta_{old}, \theta) \leq \delta
        \end{split}
\end{equation}
In the trust region, we limit our search within a region controlled by $\delta$. Mathematically, we
can prove that such region guarantees that its optimal policy will outperform the
current one until it reaches a local or global optimal.


\section{Environments}

\subsection{MountainCarContinuous-v0}
An underpowered car must climb a one-dimensional hill to reach a target. The
action (engine force applied) is allowed to be a continuous value.
The target is on top of a hill on the right-hand side of the car. If the car reaches it or goes beyond, the episode terminates.
\\
On the left-hand side, there is another hill. Climbing this hill can be used to
gain potential energy and accelerate towards the target. On top of this second
hill, the car cannot go further than a position equal to -1, as if there was a
wall. Hitting this limit does not generate a penalty.
\\
The observations are CarPosition and CarVelocity and the only action permits to
push the car on the left (negative values) or on the right (negative values).
\\
Reward is 100 for reaching the target of the hill on the right hand side, 
minus the squared sum of actions from start to goal.

This reward function raises an \textbf{exploration challenge}, because if the agent does
not reach the target soon enough, it will figure out that it is better not to move, 
and won't find the target anymore.
\\
To consider the problem solved, the reward should be over 90.


\subsection{LunarLanderContinuous-v0}
A lunar lander is a spacecraft designed to land on the Moon. Landing pad is always at coordinates
(0,0). Reward for moving from the top of the screen to landing pad and zero speed is about 100-140
points. If lander moves away from landing pad it loses reward back. Episode finishes if the lander
crashes or comes to rest, receiving additional -100 or +100 points. Each leg ground contact is +10.
Firing main engine is -0.3 points each frame (fuel is infinite). Solved is 200 points. Action is two
real values vector from -1 to +1. First controls main engine, [-1,0] off, [0,1] throttle from 50\%
to 100\% power, Second value [-1.0,-0.5] fire left engine, [0.5,1.0] fire right engine, [0.5,0.5]
off.
\begin{figure}[h!]
        \centering
        \begin{subfigure}[b]{0.4\linewidth}
                \includegraphics[width=\linewidth]{mountain_screen}
                \caption{}
        \end{subfigure}
        \begin{subfigure}[b]{0.4\linewidth}
                \includegraphics[width=\linewidth]{lunar_screen}
                \caption{}
        \end{subfigure}

\caption{ Both the environments just described. (a) MountainCarContinuous-v0 when reaching the goal position.(b) LunarLanderContinuous-v2 when reaching the landing pad.}
\label{fig:screens}
\end{figure}



\section{Results}
We should choose delta and alpha carefully to permits to the algorithm to avoid steps too large and decide what e > 10 circa.


\end{document}
